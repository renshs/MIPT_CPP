\documentclass{article}
\usepackage[utf8x]{inputenc}
\usepackage{ucs}
\usepackage{amsmath} 
\usepackage{amsfonts}
\usepackage{upgreek}
\usepackage[english,russian]{babel}
\usepackage{graphicx}
\usepackage{float}
\usepackage{textcomp}
\usepackage{hyperref}
\usepackage{geometry}
  \geometry{left=2cm}
  \geometry{right=1.5cm}
  \geometry{top=1cm}
  \geometry{bottom=2cm}
\usepackage{tikz}
\usepackage{ccaption}
\usepackage{multicol}

\usepackage{listings}
%\setlength{\columnsep}{1.5cm}
%\setlength{\columnseprule}{0.2pt}


\begin{document}
\pagenumbering{gobble}

\lstset{
  language=C++,                % choose the language of the code
  basicstyle=\linespread{1.1}\ttfamily,
  columns=fixed,
  fontadjust=true,
  basewidth=0.5em,
  keywordstyle=\color{blue}\bfseries,
  commentstyle=\color{gray},
  stringstyle=\ttfamily\color{orange!50!black},
  showstringspaces=false,
  %numbers=false,                   % where to put the line-numbers
  numbersep=5pt,
  numberstyle=\tiny\color{black},
  numberfirstline=true,
  stepnumber=1,                   % the step between two line-numbers.        
  numbersep=10pt,                  % how far the line-numbers are from the code
  backgroundcolor=\color{white},  % choose the background color. You must add \usepackage{color}
  showstringspaces=false,         % underline spaces within strings
  captionpos=b,                   % sets the caption-position to bottom
  breaklines=true,                % sets automatic line breaking
  breakatwhitespace=true,         % sets if automatic breaks should only happen at whitespace
  xleftmargin=.2in,
  extendedchars=\true,
  keepspaces = true,
}
\lstset{literate=%
   *{0}{{{\color{red!20!violet}0}}}1
    {1}{{{\color{red!20!violet}1}}}1
    {2}{{{\color{red!20!violet}2}}}1
    {3}{{{\color{red!20!violet}3}}}1
    {4}{{{\color{red!20!violet}4}}}1
    {5}{{{\color{red!20!violet}5}}}1
    {6}{{{\color{red!20!violet}6}}}1
    {7}{{{\color{red!20!violet}7}}}1
    {8}{{{\color{red!20!violet}8}}}1
    {9}{{{\color{red!20!violet}9}}}1
}

\title{Семинар \#3: Шаблоны (и строки). \vspace{-5ex}}\date{}\maketitle


\section*{Часть 1: Строки в C++}
На прошлом занятии мы писали свой класс для строк. Но в языке \texttt{C++} есть библиотека \texttt{string} (не путать с библиотекой \texttt{string.h} из языка \texttt{C}), которая предоставляет класс строк \texttt{std::string}. В отличии от строк языка C, которые являются просто массивами элеметов типа \texttt{char}, строки в языке C++ являются классом. Работать с ними гораздо проще и удобней, чем со строкам в языке \texttt{C}.
\begin{multicols}{2}
\textbf{Строки в \texttt{C}}
\begin{lstlisting}
#include <stdio.h>
#include <string.h> 

int main () 
{
    char a[10] = "Deus";
    char b[10];
    strcpy(b, "machina");
    
    char c[20];
    strcpy(c, a);
    strcat(c, " ex ");
    strcat(c, b);
    printf("%s\n", c);
    
    if (strcmp(b, a) > 0)
        printf("b is greater\n");
}
\end{lstlisting}
\vfill\null
\columnbreak
\textbf{Строки в \texttt{C++}}
\begin{lstlisting}
#include <iostream>
#include <string> 
using std::cout, std::endl;

int main() 
{
    std::string a {"Deus"};
    std::string b;
    b = "machina";
    
    std::string c = a + " ex " + b;
    cout << c << endl;
    
    if (b > a) 
        cout << "b is greater" << endl;
}
\end{lstlisting}
\end{multicols}


\newpage
\section*{Часть 2: Шаблонные функции}

\begin{lstlisting}
#include <iostream>
#include <string>
using std::cout, std::endl;

template <class T>
T getMax (T a, T b) 
{
    if (a > b)
    	return a;
    else 
    	return b;
}

int main () 
{
    int a = 5, b = 6;
    cout << getMax<int>(a, b) << endl;
    
    long long n = 9645634567, m = 7356735634;
    cout << getMax(n, m) << endl;
    
    cout << getMax(4.6, 5.3) << endl;
    
    std::string s1 = "deus"; 
    std::string s2 = "machina";
    cout << getMax(s1, s2) << endl;
}
\end{lstlisting}

\subsection*{Задачи}
\begin{itemize}
\item Написать шаблонную функцию \texttt{T triple(T x)}, которая увеличивает переменную в 3 раза. Проверить её на переменных типа \texttt{int}, \texttt{float}, \texttt{Complex}, \texttt{std::string}. Реализацию класса \texttt{Complex} можно найти в файле \texttt{complex.h}.
\begin{lstlisting}
cout << triple<int>(5) << endl;
    
std::string s = "Hello"
cout << triple(s) << endl;  // HelloHelloHello
\end{lstlisting}
\item Написать шаблонную функцию \texttt{T sum(T arr[], int size)}, которая возвращает сумму массива переменных. Проверить её на переменных типа \texttt{int}, \texttt{float}, \texttt{Complex}, \texttt{std::string}.
\begin{lstlisting}
int numbers[] = {4, 8, 15, 16, 23, 42};
cout << sum<int>(numbers, 6) << endl;
    
std::string words[] = {"Deus", "Ex", "Machina"};
cout << sum(words, 3) << endl;   // DeusExMachina
\end{lstlisting}
\end{itemize}

\newpage

\section*{Часть 3: Шаблонные классы}
Можно создавать не только шаблонные функции, но и шаблонные классы. Например, создадим шаблонный класс статического массива:
\begin{lstlisting}
#include <iostream>
using std::cout, std::endl, std::size_t;

template <typename T, size_t size>
class Array 
{
private:
    T data[size];

public:
    T& operator[](size_t id) 
    {
        return data[id];
    }
};

int main() 
{
    Array<int, 10> numbers;
    for (int i = 0; i < 10; ++i)
        numbers[i] = rand() % 100;

    cout << numbers[1] << endl;
}
\end{lstlisting}

\subsection*{Задачи}
\begin{itemize}
\item Измените оператор взятия по индексу так, чтобы при выходе за пределы массива программа выдавала ошибку и завершалась.
\item Добавьте метод \texttt{reverse}, который будет обращать статический массив \texttt{Array}. Протестировать этот метод на \texttt{Array} разного типа.
\item Добавьте метод \texttt{sort}, который будет сортировать статический массив \texttt{Array} (для простоты используйте простую $O(n^2)$ сортировку).
\item Перегрузите операторы больше и меньше для этого массива (сравнение в лексиграфическом порядке). Теперь можно будет отсортировать массив массивов.

\end{itemize}


\newpage
\section*{Часть 4: Основы std::vector}
\texttt{std::vector} - это удобный динамический массив \texttt{C++} (\texttt{vector} - не совсем удачное название, так как можно подумать, что этот массив имеет какое-то отношение к математическим векторам, но это не так). Он хранит все элементы в куче и автоматически увеличивается в размерах, если нужно. Это шаблонный класс и может хранить в себе почти что угодно. Для работы с ним в стандартной библиотеки языка \texttt{C++} уже написано множество методов и функций.\\

Для добавления новых элементов в вектор используйте метод \texttt{push\_back}. Этот метод добавит элемент в конец и увеличит размер вектора. Метод \texttt{size} возвращает размер вектора, а метод \texttt{capacity} -- его вместимость. Функции вектор передаётся также, как и другие объекты.
\begin{lstlisting}
#include <iostream>
#include <vector>
using std::cout, std::endl;

int main () 
{
    std::vector<int> v = {54, 62, 12, 97, 41, 6, 73};
    cout << v[1] << endl;
    
    v.push_back(44);
    cout << "Size = " << v.size() << " Capacity = " << v.capacity() << endl;
    
    for (int i = 0; i < v.size(); i++)
    	cout << v[i] << ' ';

    cout << endl;
}
\end{lstlisting}


      

\begin{itemize}
\item \textbf{Размер и вместимость:} Проверьте как работает автоматическое расширение вектора. Для этого создайте пустой вектор и заполните его числами от 0 до 300 (используйте \texttt{push\_back}). При этом на каждом шаге печатайте размер вектора и его вместимость.
\item \textbf{Reserve:} Постоянные расширения вектора могут быть очень трудозатратны. Используйте метод \texttt{reserve}, чтобы расширить вектор до значения 300 перед добавлением элементов. Проверьте как будет меняться размер и вместимость вектора в этом случае.
\begin{lstlisting}
v.reserve(300);
\end{lstlisting}
\item \textbf{Вектор строк:} Создадим следующий вектор строк:
\begin{lstlisting}
std::vector<std::string> animals = {"Cat", "Dog", "Bison", "Rabbit", "Spider", "Wolf", "Turkey", "Lion", "Pig", "Snake", "Shark", "Bird", "Fish"};
\end{lstlisting}
\begin{itemize}
\item Напечатайте этот вектор на экран.
\item Напечатайте только тех животных, которые начинаются на букву \texttt{S}.
\item Напишите функцию, \texttt{print\_by\_letter} которая принимает на вход вектор строк и один символ. Она должна печатать все строки, начинающиеся на этот символ.
\item Написать функцию \texttt{get\_by\_letter}, которая принимает на вход вектор строк и один символ. Эта функция должна должна возвращать вектор строки слов, которые начинаются на соответствующий символ. Для этого внутри функции вы должны создать новый вектор, заполнить его нужными строками и вернуть. Проверить правильность работы функции, вызвав её в функции main и напечатав результат.
\item Написать функцию \texttt{change\_by\_letter}, которая принимает на вход вектор строк и один символ. Функция должна заменять все слова, начинающиеся на этот символ на слово \texttt{``Animal''}.
\end{itemize}

\item \textbf{Шаблоны + векторы:} Написать шаблонную функцию \texttt{T sum(const std::vector<T>\& vec)}, которая возвращает сумму вектора переменных. Проверить её на переменных типа \texttt{int}, \texttt{float}, \texttt{Complex}, \texttt{std::string}.
\end{itemize}


\iffalse
Одна из важных идей \texttt{vector} и других контейнеров \texttt{C++} - это идея об итераторах. Итератор это специальный обьект созданный для удобства работы с вектором и другими контейнерами. \texttt{vector} имеет сложенный тип итератора \texttt{vector<T>::iterator} и специальные методы \texttt{begin} и \texttt{end}.
\begin{itemize}
\item[--] \texttt{std::vector<int>::iterator it} -- объявление итератора вектора из \texttt{int}-ов.
\item[--] \texttt{v.begin()} -- это итератор, который указывает на первый элемент
\item[--] \texttt{v.end()} -- это итератор, который указывает на элемент следующий за последним
\item[--] \texttt{*it} -- перегруженный оператор \texttt{*} возвращает ссылку на тот элемент, куда указывает итератор.
\item[--] \texttt{it++} -- перегруженный оператор \texttt{++}. Итератор теперь указывает на следующий элемент.
\end{itemize}
Стандартные 
   
    
        // Для сортировки вектора можно использовать функцию std::sort из библиотеки algorithm
        
          // Для поиска элемента в векторе нужно использовать стандартную функцию std::find из
    // библиотеки algorithm. Эта функция вернёт итератор на элемент. Если элемента
    // в векторе нет, то функция вернёт v.end()

\begin{lstlisting}
#include <iostream>
#include <string>
#include <vector>
#include <algorithm>
using namespace std;

int main () {
    std::vector<int> v = {54, 62, 12, 97, 41, 6, 73};
    cout << v[1] << endl;
    
    for (std::vector<int>::iterator it = v.begin(); it != v.end(); ++it) {
        cout << *it << " ";
    }
    cout << *it;
    
    if (std::find(v.begin(), v.end(), 55) != v.end())
    	cout << "Element found" << endl;
    else
    	cout << "Element not found" << endl;
    	
   	std::sort(v.begin(), v.end());
}
\end{lstlisting}

\begin{itemize}
\item \textbf{Обратить вектор:} Используйте функцию \texttt{reverse} из библиотеки \texttt{algorithm}, чтобы обратить вектор \texttt{v}.
\item \textbf{Вектор в функции:} Написать функцию \texttt{square\_vec}, которая принимает на вход вектор чисел типа \texttt{int} и возводит все числа вектора в квадрат.

\item \textbf{Вектор строк:} Создадим следующий вектор строк:
\begin{lstlisting}
std::vector<std::string> animals = {"Cat", "Dog", "Bison", "Dolphin", "Eagle", "Pony", "Ape", "Lobster", "Monkey", "Cow", "Deer", "Duck", "Rabbit", "Spider", "Wolf", "Turkey", "Lion", "Pig", "Snake", "Shark", "Bird", "Fish", "Chicken", "Horse"};
\end{lstlisting}
\begin{itemize}
\item Напечатайте этот вектор на экран.
\item Отсортируйте этот вектор и напечатайте его.
\item Обратите этот вектор и напечатайте его.
\item Напечатайте только тех животных, которые начинаются на букву S.
\item Написать функцию get\_first\_letter, которая принимает на вход вектор строк и один символ. Эта функция должна должна возвращать вектор строки слов, которые начинаются на соответствующий символ. Для этого внутри функции вы должны создать новый вектор, заполнить его нужными строками и вернуть. Проверить правильность работы функции, вызвав её в функции main и напечатав результат.
\end{itemize}
\end{itemize}
\fi


\newpage
\section*{Часть 5: Создаём свой динамический массив}

Динамический массив - массив, который сам расширяется при добавлении в него элементов. Он по умолчанию реализован в разных языках программирования. В частности, в языке \texttt{C++} это шаблонный класс \texttt{std::vector}. Для работы с ним нужно подключить библиотеку \texttt{<vector>}. Но в этом задании мы рассмотрим поэтапное создание своего динамического массива. Исходный код -- в папке \texttt{handmade\_dynarray}.
\subsection*{Шаг 0: Динамический массив на языке C}
В файле \texttt{0handmade\_dynarray.c} содержится исходный код для динамического массива на языке \texttt{C}. Такой мы писали в прошлом семестре, когда реализовывали стек на основе динамического массива. Также там написаны функции для работы с этим динамическим массивом. Функция \texttt{dynarray\_push\_back} -- добавляет элемент в конец массива. Обратите внимание, что для хранения размеров и индексов массива используется специальный целочисленный тип \texttt{size\_t}. Это специальный тип, который задаётся в стандартных библиотеках \texttt{C} и \texttt{C++} для хранения индексов. Обычно это просто синоним типа \texttt{unsigned int} или \texttt{unsigned long}.\\
Теперь будем поэтапно переписывать эту структуру данных с языка \texttt{C} на язык \texttt{C++}.
\subsection*{Шаг 1: Инкапсуляция}
Сначала нужно все функции для работы с динамическим массивом сделать методами класса \texttt{Dynarray}. К примеру функция:
\begin{lstlisting}
void dynarray_push_back(Dynarray* pd, Data x)
\end{lstlisting}
переходит в метод класса:
\begin{lstlisting}
void push_back(Data x)
\end{lstlisting}
\subsection*{Шаг 2: new / delete}
В языке \texttt{C++} следует всегда предпочить операторы \texttt{new/delete} функциям \texttt{malloc/free}. Поэтому на этом шаге мы поменяем все \texttt{malloc}-и на \texttt{new}, а \texttt{free} изменим на \texttt{delete}. Аналога \texttt{realloc} нет, поэтому просто сами перевыделяем память. Чтобы проверить \texttt{malloc} на правильность работы нужно сравнить его возращаемое значение с нулём. Проверка \texttt{new} на правильность работы выполняется с помощью исключений. Пока мы эту тему не прошли, так что просто ничего не проверяем.
\\
Также в этой части мы меняем все вызовы \texttt{printf} на \texttt{std::cout <{}<}.

\subsection*{Шаг 3: Конструкторы и деструктор.}
Вызов функций \texttt{init} и \texttt{destroy} при каждом создании/удалении объекта кажется не очень хорошей идеей. Если программист забудет вызвать их, то в программе возникнет ошибка или утечка памяти. Эти функции должны быть частью процесса создания/удаления объекта и должны вызываться автоматически. Перепишем эти функции в конструктор \texttt{Dynarray} и деструктор \texttt{$\sim$Dynarray} соответственно.

\subsection*{Шаг 4: Шаблоны.}
В качестве хранимого типа мы используем \texttt{Data}, который задаём с помощью \texttt{typedef}:
\begin{lstlisting}
typedef int Data;
\end{lstlisting}
Таким образом, можно изменять тип данных в массиве, но нельзя, например, создать 2 динамических массива с разными типами данных в одной программе. Используем шаблоны, чтобы добиться нужного результата.\\
\subsection*{Шаг 5: private / public.}
Программист, который будет пользоваться текущей реализации нашего массива, может легко его сломать. Например так:
\begin{lstlisting}
Dynarray<int> a;
a.size = 100000;
\end{lstlisting}
Чтобы минимизировать количество ошибок, которые могут возникнуть при работе с нашим классом, скроем поля, изменение которых может всё поломать (то есть все поля). Поля \texttt{size}, \texttt{capacity} и \texttt{values} помещаем в \texttt{private}. Так как мы всё-таки хотим дать программисту возможность знать эти значения, введём публичные методы \texttt{get\_size()} и \texttt{get\_capacity()}. Для работы с элементами массива введём функцию \texttt{at}, которая будет работать как \texttt{operator[]}, но проверять входной индекс на правильность.

\subsection*{Шаг 6: Оператор присваивания (operator=).}
Если не перегрузить оператор присваивания, то компилятор автоматически создаст свой (который будет просто копировать значения всех полей). В нашем случае это очень плохо, потому что при присваивании будет просто копироваться значение указателя \texttt{values}, а не сами элементы выделенные в динамической памяти. \\
Одна тонкость, которую нужно учесть при перегрузке этого оператора -- это случай \texttt{a = a}, то есть когда элемент присваивается самому себе.

\subsection*{Шаг 7: \texttt{initializer\_list} конструктор.}
В текущей реализации нельзя инициализировать значения нашего динамического массива также как мы делали с обычным массивом. Вот так:
\begin{lstlisting}
Dynarray<int> a = {4, 8, 15, 16, 23, 42};
\end{lstlisting}
Чтобы добавить такую возможность в наш класс нужно добавить конструктор, который будет принимать специальный объект типа \texttt{std::initializer\_list<T>}. Для копирования элементов из этого объекта в наш массив используем стандартную функцию \texttt{std::copy}.

\subsection*{Шаг 8: Итераторы.}
Для добавления использования итераторов нужно добавить вложенный класс итератора \texttt{iterator} и методы \texttt{begin} и \texttt{end}, которые возвращают итераторы на первый элемент и элемент, следующий за последним.
\begin{lstlisting}
Dynarray<string> a = {"Cat", "Dog", "Nutria", "Echidna", "Turtle", "Coati"};
for (Dynarray<string>::iterator it = a.begin(); it != a.end(); ++it)
	cout << *it << endl;
\end{lstlisting}
Итератор -- это объект особого типа с операцией унарная звёздочка (\texttt{operator*}) и с возможностью прибавлять/удалять целые числа. В нашем случае это просто указатель. Однако ничто не мешает создать свой класс для итератора и перегрузить соответствующие операторы.


\iffalse
\newpage
\subsection*{Часть 4: std::set}
\begin{lstlisting}
#include <iostream>
#include <string>
#include <set>
#include <algorithm>
using std::cout;
using std::endl;

int main () 
{
    // Множество это контейнер для быстрого добавления, удаления и поиска
    // Под капотом это бинарное дерево поиска с балансировкой
    // Соответственно, все операции выполняются за O(log(n))
    std::set<int> s = {54, 62, 12, 97, 41, 6, 73};
    // Доступ по индексу не работает
    // cout << s[1] << endl;
    
    //  Для добавления новых элементов в множество используйте insert
    s.insert(44);
    
    // Для поиска элементов используйте find
    // Функция find возвращает итератор
    // Если элемента в множестве нет, то функция вернёт s.end()
    std::set<int>::iterator it = s.find(20);
    if (it != s.end())
        cout << "Element is found" << endl;
    else
        cout << "Element isn't found" << endl;
		
    // Для удаления используйте функцию erase
    s.erase (s.find(41));

    // Для прохода по множеству используйте итератор
    for (std::set<int>::iterator it = s.begin(); it != s.end(); ++it)
        cout << *it << " ";
    cout << endl;
    // Обратите внимание, что множество set всегда отсортировано
    // Независимо от того как вы ложили туда элементы
    // Это работает так как set является бинарным деревом поиска
    
    // В STL есть другой контейнер std::unordered_set
    // Он работает похоже на std::set, но реализован не с помощью
    // бинарного дерева поиска, а с помощью хеш - таблицы
    // Элементы в нём не отсортированы, но зато все операции выполняются быстрее
    // O(log(N)) для std::set  и  O(1) в среднем для std::unordered_set
}
\end{lstlisting}

\begin{enumerate}
\item \textbf{Множество строк:} Дано следующее множество строк
\begin{lstlisting}
std::set<std::string> animals = {"Cat", "Dog", "Bison", "Dolphin", "Eagle", "Pony", "Ape", "Lobster", "Monkey", "Cow", "Deer", "Duck", "Rabbit", "Spider", "Wolf", "Turkey", "Lion", "Pig", "Snake", "Shark", "Bird", "Fish", "Chicken", "Horse"};
\end{lstlisting}
\begin{itemize}
\item Добавьте в множество новый элемент ``Hippo''.
\item Напечатайте все элементы множества.
\item Удалите строку ``Monkey''
\item Напечатайте только тех животных, которые начинаются на букву S.
\item Измените вашу структуру данных с set на unordered\_set.
\end{itemize}


\item \textbf{Особые числа:} В папке \texttt{generate\_special\_numbers} лежит файл \texttt{special\_numbers.cpp} с особыми числами. Этот набор чисел сгенерирован особым образом. В нем есть одно уникальное число, которое встречается всего 1 раз. Все остальные числа встречаются по 2 раза. Ваша задача - найти это число. Как считывать числа можно посмотреть в файле \texttt{read\_numbers.cpp}. Алгоритм решения этой задачи.
\begin{itemize}
\item Создайте множество.
\item Считывайте числа и проверяйте есть ли такое число в множестве.
\item Если такого числа в множестве нет, то его нужно добавить.
\item Если такое число в множестве есть, то нужно его из множества удалить.
\item В конце в множестве должно остаться только одно уникальное число.
\end{itemize}
Почему для решения этой задачи нужно использовать множество, а не вектор?

\end{enumerate}
\fi

\end{document}